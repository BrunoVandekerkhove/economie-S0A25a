\section{Beschrijvende Macro-Economie}

\entrystyled{micro-economie}{Micro-economie} is achter de rug. Daar ging het eerder over het gedrag van individuele economische agenten (vandaar `\textit{micro}') en het bereiken van het marktevenwicht. We gaan nu over naar de macro-economie. De \entry{macro-economie} is in feite de studie van \'e\'en grote \entry{marktfaling} : de faling van de arbeidsmarkt. Het is de theoretische grondslag van de begrippen die we in wat volgt gaan bespreken.

\subsection{Nationale Boekhouding}

\subsubsection{Het Ontstaan van de Macro-Economie}

Macro-economie is een recente discipline, ontstaan tijdens de depressie van de jaren `30, toen er veel werkloosheid was. Alles ging er dan op achteruit. Internationale handel verminderde drastisch, er was \entry{deflatie}, ... en het was in die context dat er alternatieven voor de democratie (zoals het fascisme en het communisme) naar voren sprongen.\\

\par De traditionele visie van de hoge werkloosheid was de marktvisie ; men verklaarde de werkloosheid door te stellen dat de lonen te hoog waren. Werkloosheid impliceert immers een \entry{aanbodoverschot} aan arbeid. Dus ging men er van uit dat de lonen automatisch zouden dalen, waarna de \entry{werkloosheidsgraad} zou dalen.\\

\par Keynes\footnote{John Maynard Keynes (1883-1946) was een Brits econoom.} was niet akkoord met deze traditionele opvatting. Hij vond dat men niet mocht wachten op het dalen van de lonen, omdat er \entry{loonstarheid} was. Lonen kunnen niet zo maar veranderen!
\par Nee, Keynes was van mening dat het probleem niet de hoge lonen maar de lage vraag naar goederen en arbeid was. De \entry{aggregatieve vraag} moest gestimuleerd worden (waardoor de vraagcurve naar rechts schuift).\\

\par \textit{Hoe kan men de aggregatieve vraag aanzwengelen?} Door overheidsbestedingen te doen stijgen. Keynes stelde voor om dit te doen zonder tegelijkertijd de belastingen te doen stijgen. Hij was van mening dat een overheidsdeficit soms nodig is (voorheen zag men de rol van de overheid vaak als die van de `goede huisvader' die enkel uitgeeft wat hij verdient).\\

\par Keynes kon zijn theorie niet terugvinden in de cijfers ; in die jaren bestond immers geen \entry{nationale boekhouding}. Het begrip `\entrystyled{bbp}{BBP}' was zelfs nog niet uitgevonden! 
\par De theorie ging dus vooraf aan de empirie, wat een helikopterperspectief noodzaakte. Men had met name aggregaten en indexen (algemene prijsniveau's) nodig.

\subsubsection{Het BBP}

Men definieerde het \entrystyled{bbp}{Bruto Binnenlands Product (BBP)} eerder al in hoofdstuk \ref{sec:h1bbp} als \textit{de som van de gedurende \'e\'en jaar door alle binnenlandse productie-eenheden toegevoegde waarden}. Het is een aggregaat en wordt dus in waarde (geld) uitgedrukt.
\par `\textit{Bruto}' houdt in dat men de productie overschat. Men houdt bijvoorbeeld geen rekening met de waardevermindering van de \entry{kapitaalstock} tijdens het afgelopen jaar. Deze kapitaalstock (zoals machines, bruggen, wegen, ...) gaat over de door de mens geproduceerde duurzame productiemiddelen. En die verliezen geleidelijk hun waarde. Men noemt dat de \entry{depreciatie} of \entry{afschrijving}.
\par `\textit{Binnenlands}' omdat het een territoriaal concept is (in tegenstelling tot het \entrystyled{bni}{Bruto Nationaal Inkomen (BNI)}). Het gaat over de productie binnen de landsgrenzen, voornamelijk door de overheid en de bedrijven.\\

\par Het \entrystyled{bbp}{BBP} is een \entry{stroomveranderlijke}. Dat wil zeggen dat het betrekking heeft op een periode.

\subsubsection{De Drie Wijzen}

Zoals men in hoofdstuk \ref{sec:h1bbp} besprak, kan het BBP benaderd worden aan de hand van de productie (de toegevoegde waarden), de inkomens of de bestedingen. We gaan daar dieper op in. \\

\par - \textit{Productiebenadering} : de toegevoegde waarde is gelijk aan de omzet min de \entrystyled{intermediaire goederen}{intermediaire inputs} (en dus niet gewoon de som van de waarde van eindproducten, of men doet aan dubbeltelling). Bij de productie wordt tarwe bijvoorbeeld omgezet in meel, waarvan broden worden gemaakt. De toegevoegde waarde is dan gelijk aan de waarde van de broden (de output) minus de waarde van het gebruikte meel (de input).\\

\begin{leftbar}\noindent\textit{BBP = Waarde finale producten - Waarde intermediaire producten.}\end{leftbar}

\par - \textit{Inkomensbenadering} : de gerealiseerde toegevoegde waarden worden als inkomen verdeeld onder de eigenaars van de productiefactoren (het arbeid en het kapitaal). De factorvergoedingen kunnen dus gebruikt worden om het BBP te benaderen. 
\par Het BBP is dan de som van de lonen (wat je krijgt in ruil voor je werk) en de vermogensinkomens (wat je krijgt omdat je vermogen tot beschikking stelt). Omdat men werkt met marktprijzen moet er echter ook rekening gehouden worden met belastingen, zoals BTW en accijnzen :

\begin{leftbar}\noindent\textit{BBP = Indirecte belastingen + Inkomen uit arbeid + Bruto exploitatieoverschot van bedrijven \textcolor{gray}{(= vermogensinkomen = alles wat niet naar de arbeiders gaat)} + Gemengd inkomen \textcolor{gray}{(= de inkomens waarvoor het onderscheid tussen arbeids- en kapitaalinkomen niet kan gemaakt worden, zoals die van de zelfstandigen)}.}\end{leftbar}

\par - \textit{Bestedingsbenadering} : de geproduceerde goederen worden gekocht in de winkel. Het BBP kan daarom ook benaderd worden op basis van de bestedingen. Om hier ook dubbeltelling te vermijden mag men enkel rekening houden met finale goederen.
\par Bestedingen worden zowel door de consument als door de staat gedaan. Dit noemt men respectievelijk de private - en de overheids-consumptie.
\par Bij de bestedingsbenadering moet men ook rekening houden met de \entrystyled{investering}{investeringen}, d. i. de stijging van de kapitaalstock. Het zijn voornamelijk de bedrijven die investeren door bijvoorbeeld machines en gebouwen te kopen. Gemeentes, de overheid en de gezinnen investeren echter ook. Men kijkt enkel naar de bruto investeringen, dat wil zeggen, men houdt geen rekening met de vermindering van de waarde van de kapitaalstock. Het gaat immers over het \textit{bruto} binnenlands product.
\par De consumptie en de investeringen samen noemt men de \entry{binnenlandse vraag}. En deze binnenlandse vraag valt \textit{niet} samen met het BBP, omdat men uiteindelijk ook verkoopt aan (en koopt van) het buitenland. Men moet dus de export - en import van goederen in rekening brengen. Het verschil tussen de export en de import noemt men de \entry{handelsbalans}\footnote{De handelsbalans gaat \textit{niet} over winst.} of de \textit{netto export}. De som van de \entry{binnenlandse vraag} en de \entry{handelsbalans} geeft ons een benadering voor het \entrystyled{bbp}{BBP} :

\begin{leftbar}\noindent\textit{BBP = Private consumptie (C) + Overheidsconsumptie (G) + Investeringen (I) + Export (E) - Import (Z).}\end{leftbar}

Merk op dat $BBP-E+Z=C+G+I$. Als er een deficit is op de handelsbalans, dan is de binnenlandse vraag dus groter dan het $BBP$. Als dit jarenlang geldt, dan is er een probleem, en moet men het deficit op \'e\'en of andere manier wegwerken. \'Of door de binnenlandse vraag te verlagen, \'of door het $BBP$ te verhogen. Het eerste is gemakkelijk en gaat gepaard met een bezuinigingspolitiek (of \entry{austeriteitspolitiek}). De bezuiniging leidt soms tot een verlaagde productie. Het alternatief - het verhogen van de productie of het $BBP$ - is dus nuttiger.

\subsubsection{Het BBP in Belgi\"e}

Figuur \ref{fig:h5bbpbel} geeft het BBP (per sector) in Belgi\"e. Het is duidelijk dat de dienstensector alleen maar gegroeid is, dat er desindustrialisatie is, en dat de landbouwsector ook gekrompen is.

\begin{figure}[H]
\small\centering\captionsetup{justification=centering,margin=2cm}
\begin{tikzpicture}
\begin{axis}[axis lines=left,axis line style=gray,/pgf/number format/.cd, use comma, 1000 sep={}, width=0.5\linewidth, legend cell align=left,legend style={at={(axis cs:2010,40)},anchor=south west,draw=none},ymin=0,ymax=100,ylabel={\% toegevoegde waarden}]
\addplot[blue] table [x=Jaar, y=Diensten, col sep=comma] {Data/H5-BBP.csv};
\addplot[red]  table [x=Jaar, y={Industrie}, col sep=comma] {Data/H5-BBP.csv};
\addplot[black,]  table [x=Jaar, y=Landbouw, col sep=comma] {Data/H5-BBP.csv};
\addlegendentry{Diensten}
\addlegendentry{Industrie}
\addlegendentry{Landbouw}
\end{axis}
\end{tikzpicture}
\caption{De productiebenadering van het BBP in Belgi\"e}
\label{fig:h5bbpbel}
\end{figure}

Zoals te zien is in figuur \ref{fig:h5bbpwereld} is de landbouwsector groter in de ontwikkelingslanden. Merk op dat zo'n 76\% van de bevolking in Congo in de landbouw werkt, maar de toegevoegde waarde toch laag is. Dat komt omdat de productiviteit in de landbouwsector laag is.

\begin{figure}[H]
\centering
\begin{tikzpicture}
\begin{axis}[y dir=reverse,xbar stacked,axis x line=bottom,ytick=data,yticklabels from table={Data/H5-BBP2.csv}{Land},table/col sep=comma]
\addplot [fill=barblue!50] table [x=Landbouw, meta=Land,y expr=\coordindex,col sep=comma] {Data/H5-BBP2.csv}; 
\addplot [fill=barred!50] table [x=Industrie, meta=Land,y expr=\coordindex,col sep=comma] {Data/H5-BBP2.csv};
\addplot [fill=bargreen!50] table [x=Diensten, meta=Land,y expr=\coordindex,col sep=comma] {Data/H5-BBP2.csv};
\end{axis}
\end{tikzpicture}
\caption{Het BBP ontleed voor een aantal landen (in \% toegevoegde waarde)}
\label{fig:h5bbpwereld}
\end{figure}

In Belgi\"e daalt het loonaandeel (het deel van het BBP dat naar de arbeid gaat\footnote{De gemengde inkomens worden hier niet als loonaandeel gerekend, wat dus een onderschatting geeft.}), wat wil zeggen dat het vermogensaandeel stijgt (figuur \ref{fig:h5loonaan}). Een belangrijke verklaring hiervoor is de globalisering. In die context is het gemakkelijker kapitaal te bewegen dan arbeid te bewegen (het is gemakkelijker in het buitenland te investeren dan te verhuizen naar het buitenland), wat een machtspositie geeft aan werkgevers.

\begin{figure}[H]
\small\centering\captionsetup{justification=centering,margin=2cm}
\begin{tikzpicture}
\begin{axis}[axis lines=left,axis line style=gray,/pgf/number format/.cd, use comma, 1000 sep={}, width=0.5\linewidth,ymin=0,ymax=1,ytick={0.2,0.4,0.6,0.8,1.0},yticklabels={20\%,40\%,60\%,80\%,100\%},table/col sep=comma]
\addplot[blue] table [x=Jaar, y=Belgie, col sep=comma] {Data/H5-Loonaandeel.csv};
\end{axis}
\end{tikzpicture}
\caption{Het loonaandeel in Belgi\"e}
\label{fig:h5loonaan}
\end{figure}

Figuur \ref{fig:h5bbpbel2} geeft het aandeel van het BBP per bestedingscategorie doorheen de laatste decennia's weer. De private consumptie is relatief stabiel. Als de aggregatieve vraag daalt, dan heeft dit meestal te maken met investeringen, die volatiel (minder stabiel) zijn.
\par De overheidsbestedingen stijgen doorheen de tijd. De handelsbalans is meestal positief, wat wil zeggen dat men meer exporteert dan importeert. De deficit rond 1980 was ten tijde van de oliecrisis. Toen gingen de prijzen naar omhoog, en indexeerde men tegelijkertijd de lonen. Dit zorgde voor een inflatoire spiraal (of \entry{loon-prijsspiraal}) : de lonen stijgen, de prijzen stijgen, de lonen stijgen, ... In 1982 werd de Belgische Frank dan gedevalueerd, zodat de handelsbalans weer positief werd.

\begin{figure}[H]
\small\centering\captionsetup{justification=centering,margin=2cm}
\begin{tikzpicture}
\begin{axis}[axis lines=left,axis line style=gray,/pgf/number format/.cd, use comma, width=0.5\linewidth, legend cell align=left,legend style={at={(axis cs:2010,0.4)},anchor=south west,draw=none},ymin=-0.2,ymax=0.7,ytick={0,0.2,0.4,0.6},yticklabels={0\%,20\%,40\%,60\%}]
\addplot[blue] table [x=Jaar, y=Consumptie, col sep=comma] {Data/H5-BBPBelgie.csv};
\addplot[darkgreen] table [x=Jaar, y=Investeringen, col sep=comma] {Data/H5-BBPBelgie.csv};
\addplot[red] table [x=Jaar, y=Overheidsconsumptie, col sep=comma] {Data/H5-BBPBelgie.csv};
\addplot[brown] table [x=Jaar, y=Handelsbalans, col sep=comma] {Data/H5-BBPBelgie.csv};
\addlegendentry{Consumptie}
\addlegendentry{Investeringen}
\addlegendentry{Overheidsconsumptie}
\addlegendentry{Handelsbalans}
\end{axis}
\end{tikzpicture}
\caption{Bestedingscategorie\"en in Belgi\"e}
\label{fig:h5bbpbel2}
\end{figure}

Bepaalde landen hebben een positieve netto-export, anderen een negatieve. China heeft een zeer hoge economische groei omdat het land veel investeert.\\

\par Figuur \ref{fig:h5opecon} geeft de afzonderlijke export - en import van Belgi\"e doorheen de tijd. In 2015 exporteerde en importeerde Belgi\"e meer dan 80\%! Er is dus, net als in Nederland, een \entry{open economie}. Dat heeft te maken met het feit dat beide landen klein zijn. Onder andere door effici\"entere transportmogelijkheden wordt de economie opener en opener.
\par De daling in 2009 is te weiten aan de financi\"ele crisis, die de internationale handel deed dalen.

\begin{figure}[H]
\small\centering\captionsetup{justification=centering,margin=2cm}
\begin{tikzpicture}
\begin{axis}[y dir=reverse,xbar,bar width=3pt,axis x line=bottom,ytick=data,yticklabels from table={Data/H5-Handelsbalans.csv}{Land},table/col sep=comma,/pgf/number format/fixed,ytick=data,xmin=0]
\addplot [fill=barred!50] table [x=Import, y expr=\coordindex,col sep=comma] {Data/H5-Handelsbalans.csv};
\addplot [fill=barblue!50] table [x=Export, y expr=\coordindex,col sep=comma] {Data/H5-Handelsbalans.csv}; 
\addlegendentry{Import}
\addlegendentry{Export}
\end{axis}
\end{tikzpicture}
\caption{Openheid van de economie in 20112 : evolutie en internationale vergelijking (export als \% van het BBP)}
\label{fig:h5opecon}
\end{figure}

\subsubsection{Van het BBP naar het NNBI}

We definieerden eerder het \entrystyled{bbp}{BBP}. Als men bij het BBP het netto \entry{factorinkomen} (of `\term{primair inkomen}') $NFIB=FIB_{in}-FIB_{uit}$ (met $FIB_{in}$ al de factorinkomens die binnenkomen uit het buitenland, en $FIB_{uit}$ de factorinkomens die naar het buitenland gaan) optelt, dan krijgt men het \entrystyled{bni}{Bruto Nationaal Inkomen (BNI)}.
\par Trekt men de \entrystyled{afschrijving}{afschrijvingen}\footnote{Merk op dat men de afschrijving niet in rekening bracht bij het BBP. Hier wel.} af van het \entrystyled{bni}{BNI}, dan heeft men het \entrystyled{nni}{Netto Nationaal Inkomen (NNI)}.
\par Telt men de netto \entrystyled{transfer}{transfers}\footnote{Voorbeelden van transfers zijn ontwikkelingshulp, lidgeld en geldzendingen van migranten naar hun land van herkomst. `\textit{Netto}' wilt uiteraard zeggen dat we wat we geven aftrekken van wat we krijgen.} (NTRA of `\term{secundair inkomen}') uit het buitenland op bij het \entrystyled{nni}{NNI}, dan bekomt men uiteindelijk het \entrystyled{nnbi}{Netto Nationaal Beschikbaar Inkomen (NNBI)}.\\

\par Laat men dit even doen voor Belgi\"e :

\begin{center}
\begin{tabular}{ll}
\entrystyled{bbp}{BBP} in miljarden euro & 410.4 \\
+ netto factorinkomens uit het buitenland (NFIB) & 0.1 \\
= \entrystyled{bni}{BNI} & 410.5 \\
$-$ afschrijvingen & 79.5 \\
= \entrystyled{nni}{NNI} & 331.0 \\
+ netto transfers uit het buitenland (NTRA) & -6.3 \\
= NNBI & 324.7
\end{tabular}
\end{center}

Voor andere landen is de ratio tussen het \entrystyled{bni}{BNI} en het \entrystyled{bbp}{BBP} de volgende :

\begin{center}
\begin{tabular}{ll}
Equatoriaal Guinea & 54.8\%  \\
Luxemburg          & 69.4\%  \\
Ierland            & 84.9\%  \\
Belgi\"e           & 100.0\% \\
Verenigde Staten   & 101.0\% \\
Lesotho            & 119.4\%
\end{tabular}
\end{center}

\par Equatoriaal Guinea heeft een BNI dat klein is ten opzichte van het BBP. Daar wordt olie ontgind, en het zijn buitenlandse bedrijven die dat doen. De productie wordt vervoerd naar het land van herkomst. De grote aanwezigheid van multinationale bedrijven impliceert gewoonlijk dat het BNI lager is dan het BBP.
\par Een ander extreem is Lesotho (een enclave in Zuid-Afrika), waar het \entrystyled{bni}{BNI} dus groter dan het \entrystyled{bbp}{BBP}. Dat heeft te maken met het feit dat er daar veel pendelaars zijn.\\

\par Hieronder zie je de ratio van netto transfers tot het \entrystyled{nnbi}{NNBI} in een aantal landen :

\begin{center}
\begin{tabular}{ll}
Liberia     & 40.1\% \\
Afghanistan & 31.9\% \\
Moldavi\"e  & 17.3\% \\
Palestina   & 9.5\%  \\
Marokko     & 9.4\%  \\
Congo       & 8.0\%  \\
Roemeni\"e  & 3.1\%  \\
Pakistan    & 1.3\%  \\
Belgi\"e    & -2.0\%
\end{tabular}
\end{center}

Landen als Liberia hebben veel ontwikkelingshulp gekregen omdat er daar conflicten waren (burgeroorlog).

\subsubsection{NNBI, Lopende Rekening \& Nationale Sparen}

We zagen eerder dat \textit{NNBI = BBP + NFIB - dep + NTra}. Omdat \textit{BBP = C + G + I + E - Z} geldt dus dat :
\begin{leftbar}
\textit{NNBI = (C + G + I\textsubscript{netto}) + (E - Z + NFIB + NTra)}
\end{leftbar}
\noindent\textit{I\textsubscript{netto}} is hier de netto investering, de investering waarvan de depreciatie is afgetrokken.
\par De tweede term (\textit{E - Z + NFIB + NTra}) noemt men de \entry{lopende rekening} (\textit{LR}). Het is een maatstaaf om de buitenlandse handel van een land uit te drukken. \\

\par Uit de vorige formule voor het \textit{NNBI} kan men afleiden dat :
\begin{leftbar}
\textit{NNBI - C - G = I\textsubscript{netto} + LR}
\end{leftbar}
Het rechterlid van deze vergelijking noemt men het \textit{nationale sparen}\index{nationale sparen} :
\begin{leftbar}
\textit{S = I\textsubscript{netto} + LR}
\end{leftbar}
\entrystyled{sparen}{Sparen} is het verschil tussen het inkomen en de consumptie, wat vervolgens wordt gebruikt om te beleggen. Men kan het nationale sparen opsplitsen :
\begin{leftbar}
\textit{(NNBI - C - T) + (T - G) = I\textsubscript{netto} + LR}
\end{leftbar}
Omdat \textit{T} het \entry{totaal beschikbaar overheidsinkomen} is, is \textit{(NNBI - C - T)} het \term{private sparen} en \textit{(T - G)} het \term{publieke sparen} : \textit{S\textsubscript{pri} + S\textsubscript{pu} = I\textsubscript{netto} + LR}. En dus :
\begin{leftbar}
\textit{S\textsubscript{pri} + S\textsubscript{pu} - LR = I\textsubscript{netto}}
\end{leftbar}
Als de lopende rekening positief is, dan kan men zeggen dat het private - \& publieke sparen de netto investering en kredietverstrekking aan het buitenland financiert. Is de lopende rekening negatief (`\textit{deficitair}'), dan zijn het de lopende rekening en het private - \& publieke sparen die de binnenlandse investeringen financieren. In het eerste geval noemt men de lopende rekening `\textit{onze} buitenlandse investeringen', in het andere geval noemt men die het `buitenlands sparen'.\\

\par De lopende rekening is uiteindelijk het verschil tussen wat binnenkomt en wat buitengaat. Het is gelijk aan het private - \& publieke sparen minus de netto investeringen :
\begin{leftbar}
\textit{LR = S\textsubscript{pri} + S\textsubscript{pu} - I\textsubscript{netto}}
\end{leftbar}
Men kan de lopende rekening dus ook bekijken in termen van sparen en investeringen, in plaats van export en import. Vanuit dit oogpunt wilt een deficitaire lopende rekening zeggen dat we schulden opbouwen in het buitenland. Is de lopende rekening positief, dan heeft het buitenland schulden bij ons\footnote{Een klassieke examenvraag : als je voorzitter van het IMF, \textit{is het dan een goed idee om voor te stellen dat elk land een positieve lopende rekening moet hebben}? Het antwoord is negatief ; dit is simpelweg niet mogelijk. Voorstellen dat de lopende rekening nul moet zijn voor ieder land zou op zijn beurt impliceren dat geen land schulden mag maken, wat nadelig is. Dat is dus ook geen goed idee.}.\\

\par Een deficitaire lopende rekening kan nuttig zijn, maar als het lang aanhoudt en de schuld opstapelt, dan kan het gebeuren dat de schuldenaar zijn schulden niet kan afbetalen. Dat gebeurde met Griekenland, dat schulden (in euro\footnote{Als een land schulden heeft \textit{in zijn eigen munt}, dan is dat niet erg, omdat die munt dan gewoon gedevalueerd kan worden.}) had.
\par Bij zo'n hoge buitenlandse schuld wordt aan het land gevraagd om de lopende rekening te normaliseren. Men kan dan de investeringen doen dalen, maar dit is een slecht idee omdat dit er op lange termijn voor zorgt dat de productiecapaciteit niet genoeg toeneemt, wat de zaken er erger op maakt.\\
\par Meer belastingen heffen (en minder uitgaven doen) is een alternatief. Daardoor vermindert men de consumptie van de private sector. Het is helaas zo dat deze consumptie vaak essentieel is (zoals gezondheidszorg).
\par Buitenlandse schuld aankaarten kan dus via het sparen en investeren.\\

\par De Belgische \entry{lopende rekening} wordt weergegeven in figuur \ref{fig:h5bellr}.

\begin{figure}[H]
\small\centering\captionsetup{justification=centering,margin=2cm}
\begin{tikzpicture}
\begin{axis}[axis lines=left,axis line style=gray,/pgf/number format/.cd, use comma, width=0.5\linewidth, legend cell align=left,legend style={at={(axis cs:2010,0.1)},anchor=south west,draw=none},ymin=-0.15,ymax=0.20,ytick={-0.1,0,0.1,0.2},yticklabels={-10\%,0\%,10\%,20\%}]
\addplot[black] table [x=Jaar, y=Private Sparen, col sep=comma] {Data/H5-BelgieLopendeRekening.csv};
\addplot[red] table [x=Jaar, y=Publieke Sparen, col sep=comma] {Data/H5-BelgieLopendeRekening.csv};
\addplot[orange] table [x=Jaar, y=Netto Investeringen, col sep=comma] {Data/H5-BelgieLopendeRekening.csv};
\addplot[blue] table [x=Jaar, y=Saldo op de Lopende Rekening, col sep=comma] {Data/H5-BelgieLopendeRekening.csv};
\addlegendentry{Private Sparen}
\addlegendentry{Publieke Sparen}
\addlegendentry{Netto Investeringen}
\addlegendentry{Saldo op de Lopende Rekening}
\end{axis}
\end{tikzpicture}
\caption{De lopende rekening van Belgi\"e (in procent van het BBP)}
\label{fig:h5bellr}
\end{figure}

Onze lopende rekening is meestal positief. In 1980 was er een dieptepunt. Men devalueerde toen de Belgische frank, wat het begrotingsdeficit verminderde. Vergelijkt men met andere landen (figuur \ref{fig:h5werlr}), dan ziet men dat de lopende rekening van de VS systematisch deficitair is ; de Amerikanen leven boven hun stand. China heeft dan weer een excedentaire lopende rekening. Een arm land financiert dus een rijk land!\\

\par Voor de VS is een deficitaire lopende rekening geen groot probleem omdat het land de dollar gebruikt, waarmee de rest van de wereld onder mekaar betaalt. Daardoor kan de VS gewoon dollars uitgeven zonder dat deze dollars terugkeren, want de landen gebruiken het onder elkaar. Daardoor is de opbouw van schuld niet zo'n probleem. Een \entry{reservemunt} hebben geeft de VS dus een voordeel.
\par Ondanks het feit dat de buitenlandse schuld die de VS in andere landen heeft hoger ligt dan de schuld die andere landen in de VS hebben, is er bovendien nog steeds een positief netto factorinkomen. Dat komt omdat de beleggingen van de VS in het buitenland meer opbrengen dan de beleggingen van het buitenland in de VS.

\begin{figure}[H]
\small\centering\captionsetup{justification=centering,margin=2cm}
\begin{tikzpicture}
\begin{axis}[axis lines=left,axis line style=gray,/pgf/number format/.cd, use comma, width=0.5\linewidth, legend cell align=left,legend style={at={(axis cs:2010,5)},anchor=south west,draw=none},ymin=-8,ymax=12]
\addplot[blue] table [x=Jaar, y=Belgie, col sep=comma] {Data/H5-WereldLopendeRekening.csv};
\addplot[red] table [x=Jaar, y=VS, col sep=comma] {Data/H5-WereldLopendeRekening.csv};
\addplot[orange] table [x=Jaar, y=China, col sep=comma] {Data/H5-WereldLopendeRekening.csv};
\addlegendentry{Belgi\"e}
\addlegendentry{VS}
\addlegendentry{China}
\end{axis}
\end{tikzpicture}
\caption{De lopende rekening van Belgi\"e, de VS en China (in procent van het BBP)}
\label{fig:h5werlr}
\end{figure}

\subsubsection{De Betalingsbalans}

Zoals eerder opgemerkt houdt een excedentaire lopende rekening in dat men tegoeden accumuleert (wat als beleggingen gezien kan worden). Een deficitaire lopende rekening impliceert dan weer dat men verplichtingen aan het buitenland accumuleert (wat als schulden gezien kan worden).\\

\par De lopende rekening wordt in rekening gebracht in de \entry{betalingsbalans}. Deze registreert alle internationale economische transacties die plaats hebben gehad gedurende een bepaalde periode. Hier gaan we later op in. 

\subsubsection{Nominale en Re\"ele Economische Groei}

Het verschil tussen \entry{nominale groei} en \entrystyled{reele groei}{re\"ele groei} werd al in hoofdstuk \ref{sec:h1bbp} besproken. Nominale groei gaat over zowel de kwantiteit als de prijsverandering. Re\"ele groei gaat enkel over kwantiteit, en is dus gelijk aan de nominale groei waarvan de inflatie werd afgetrokken. \\

\par Neem bijvoorbeeld tabel \ref{tab:h5appe}. De `\textit{lopende prijzen}' kolom geeft de nominale groei. Dat is de som van de waarde (prijs maal kwantiteit) van appelen en peren. Tussen 2009 en 2010 stijgt het nominaal \entrystyled{bbp}{BBP} dus met 20\%.
\par Tussen 2010 en 2011 stijgen kwantiteiten niet, maar is er een prijsstijging. Er is dus louter nominale groei, en geen re\"ele groei (die gegeven wordt in kolom `\textit{prijzen van 2009}').

\begin{table}[H]
\centering
\begin{tabular}{ccccccc}
 & \multicolumn{2}{c}{Kwantiteiten} & \multicolumn{2}{c}{Prijzen} & \multicolumn{2}{c}{BBP} \\
 & Appelen & Peren & Appelen & Peren & Lopende prijzen & Prijzen van 2009 \\ \hline
2009 & 10 & 5 & 10 & 15 & 175 & 175 \\
2010 & 12 & 6 & 10 & 15 & 210 & 210 \\
2011 & 12 & 6 & 12 & 18 & 252 & 210
\end{tabular}
\caption{Appelen en peren ; een voorbeeld van nominale - en re\"ele groei}
\label{tab:h5appe}
\end{table}

Om het prijseffect uit te schakelen neemt men in 2010 en 2011 de prijzen van een basisjaar (in het huidige voorbeeld is dat 2009).\\

\par Nu is het voorbeeld wel simplistisch. Wij leven in een dynamische economie, dus produceren en kopen we niet alleen appelen en peren. Er zijn bijvoorbeeld producten in, zeg maar, 2015, die nog niet bestonden in 1995. Anderzijds zijn er ook producten die verdwijnen (zoals de \textit{walkman}). We moeten dan anders te werk gaan.\\

\par Merk eerst even op dat het nominale \entrystyled{bbp}{BBP} uitgedrukt kan worden als \mbox{volgt :}
$$\sum_{i=0}^k q_i^t\cdot p_i^t$$
Hierbij is $q_i^t$ (of $p_i^t$) de kwantiteit (of prijs) van product $i$ in het jaar $t$. Het re\"ele BBP is gelijk aan :
$$\sum_{i=0}^k q_i^t\cdot p_i^0$$
Met $p_i^0$ de prijs van het product $i$ in het basisjaar (jaar `0'). \\

\par\noindent Stel nu dat het nominaal BBP in het jaar 2000 gelijk is aan :
$$BBP^{2000}=\sum_{i=1}^k p_i^{2000}\cdot q_i^{2000}$$
Het re\"eel BBP in het jaar 2001 is dan :
$$BBP^{2001}=\sum_{i=1}^k p_i^{2000}\cdot q_i^{2000}\times\text{ hoeveelheidindex }=\sum_{i=1}^k p_i^{2000}\cdot q_i^{2000}\times\frac{\sum_{i=1}^k p_i^{2000}\cdot q_i^{2001}}{\sum_{i=1}^k p_i^{2000}\cdot q_i^{2000}}$$
\noindent De kwantiteiten in het jaar 2001 worden gewaardeerd tegen de prijzen in het jaar 2000 (het basisjaar). Men heeft het over een \term{hoeveelheidindex}\footnote{Doet men dit met prijzen in plaats van hoeveelheden, dan heeft men het over een \term{prijsindex} $\frac{\sum_{i=1}^k p_i^{t}\cdot q_i^{t}}{\sum_{i=1}^k p_i^{0}\cdot q_i^{t}}$.}. Dit kan \entry{recursief} toegepast worden :
$$BBP^{2002}=\sum_{i=1}^k p_i^{2000}\cdot q_i^{2000}\times\frac{\sum_{i=1}^k p_i^{2000}\cdot q_i^{2001}}{\sum_{i=1}^k p_i^{2000}\cdot q_i^{2000}}\times\frac{\sum_{i=1}^k p_i^{2001}\cdot q_i^{2002}}{\sum_{i=1}^k p_i^{2001}\cdot q_i^{2001}}$$
$$...$$
$$BBP^{2003}=\sum_{i=1}^k p_i^{2000}\cdot q_i^{2000}\times\frac{\sum_{i=1}^k p_i^{2000}\cdot q_i^{2001}}{\sum_{i=1}^k p_i^{2000}\cdot q_i^{2000}}\times\frac{\sum_{i=1}^k p_i^{2001}\cdot q_i^{2002}}{\sum_{i=1}^k p_i^{2001}\cdot q_i^{2001}}\times\frac{\sum_{i=1}^k p_i^{2002}\cdot q_i^{2003}}{\sum_{i=1}^k p_i^{2002}\cdot q_i^{2002}}$$
\textit{Enzovoort ...} Algemeen is de hoeveelheidindex gelijk aan
$$Q_t = \frac{\sum_{i=1}^k p_i^0 q_i^t}{\sum_{i=1}^k p_i^0 q_i^0}$$
Deze schakels, die dus recursief worden gebruikt, noemen we \term{kettingindexen}. Het verschil met ons voorgaand simplistisch voorbeeld is dat we jaar na jaar tellen, zodat het aanbod van producten in de economie niet zo verschillend is.\\

\par Aan de hand van het BBP kan men de economische groei bepalen. De \textit{absolute} groei is simpelweg het verschil tussen het BBP van twee jaartallen. De \textit{relatieve} groei is gelijk aan de absolute groei gedeeld door het BBP van het beginjaar. 

\subsubsection{Jaarlijkse Gemiddelde Groei of Trendgroei}\label{sec:h5groei}

\textit{Maar hoe berekent men de jaarlijkse gemiddelde groei?} Voor de groei van een beginjaar $Y_0$ naar een ander jaar geldt :
$$Y_1=Y_0(1+g_1)$$
$$Y_2=Y_0(1+g_1)(1+g_2)=Y_0(1+g_g)^2$$
$$...$$
$$Y_n=Y_0(1+g_g)^n$$
Dus geldt voor de gemiddelde groeivoet $g_g$ :
$$(1+g_g)=\sqrt[n]{\frac{Y_n}{Y_0}}\qquad\Rightarrow\qquad g_g=\frac{Y_n}{Y_0}^{\frac{1}{n}}-1$$ 

Deze groei ziet men in het re\"eel Belgisch BBP (zie figuur \ref{fig:h5rebel}). De groei die men wil bereiken noemt men het \term{potentieel BBP} (d.i. de BBP als de groei steeds op langetermijn-gemiddel zou gebleven zijn.). De groei (en dus het \term{feitelijk BBP}) valt hier niet altijd mee samen. Het verschil tussen beide noemt men de \entry{outputkloof}.

\begin{figure}[H]
\small\centering\captionsetup{justification=centering,margin=2cm}
\begin{tikzpicture}
\begin{axis}[axis lines=left,axis line style=gray,/pgf/number format/.cd, use comma, width=0.3\linewidth,ymin=12.5,ymax=13]
\addplot[red] table [x=Jaar, y=BBP, col sep=comma] {Data/H5-ReeelBBPBelgie.csv};
\addplot[blue,dashed, samples=50, domain=1995:2015]{12.59+0.017*(x-1995)};
\end{axis}
\end{tikzpicture}
\caption{Het re\"eel Belgisch BBP (op logaritmische schaal\ref{sec:applog}, met in het blauw het potentieel BBP)}
\label{fig:h5rebel}
\end{figure}

\noindent Een negatieve outputkloof wijst eerder op \entry{conjuncturele werkloosheid}, een positieve outputkloof op \entry{inflatoire druk}.

\subsubsection{Marktwisselkoers en PPP wisselkoers}

Om het \entrystyled{bbp}{BBP} doorheen de tijd te vergelijken moet men het re\"ele BBP gebruiken. Als men vergelijkt doorheen de ruimte, dan maakt men gebruik van \entrystyled{koopkrachtpariteit (KKP)}{PPP-dollars} (koopkrachtpariteit-dollars). Dit concept vermeldden we eerder al in hoofdstuk \ref{sec:h1bbp}.
\par Zoals we toen opmerkten kan men hier niet zo maar vergelijken op basis van de wisselkoers (`\textit{hoeveel dollars is \'e\'en rupee waard}'?). Nee, men moet de prijzen van producten uit het ene land vergelijken op basis van de prijzen in het andere land.
\par Een Indisch brood nemen wij dus op op basis van de prijs van dat brood in dollar in de Verenigde Staten. We hebben dan geen \textit{basisjaar}, maar een \textit{basisland}.\\

\par Stel, in Indi\"e is het BBP gelijk aan 103.720 rupee (dit is een feitelijk cijfer). Op de wisselmarkt is 1 rupee $\frac{1}{65}^{\text{ste}}$ van een dollar waard. Tegen deze wisselkoers bedraagt het Indische BBP per capita $\frac{103.720}{65}=1596$ dollar.
\par Laten we nu overschakelen naar het BBP tegen koopkrachtpariteit. Een representatieve korf goederen koop je in Indi\"e voor 10.000 rupee. Diezelfde korf kost 550 dollar in de Verenigde Staten. In de \entry{goederenmarkt} is de koopkracht van 10.000 rupee gelijk aan 550 dollar. Hier is 1 rupee gelijk aan $\frac{10.000}{550}=\frac{1}{18}$ dollar. Op de goederenmarkt heeft de rupee dus meer koopkracht dan op de \entry{wisselmarkt}.
\par Het Indisch BBP per capita wordt nu $\frac{103.720}{18}=5708$, en ligt dus drie maal hoger. In Indi\"e liggen de prijzen dus drie maal lager dan in de Verenigde Staten. De PPP-wisselkoers geeft met andere woorden de prijsverhouding weer.\\

\par In tabel \ref{tab:h5bbpcap} worden de BBP's per capita vergeleken. Uiteraard is het BBP voor de Verenigde Staten dezelfde op de wisselmarkt als op de goederenmarkt. We gebruiken dit land immers als \term{basisland}.
\par De PPP-dollars liggen in Belgi\"e lager omdat het leven hier duurder is. Voor de andere landen (of regio) is dit omgekeerd.

\begin{table}[H]
\centering
\begin{tabular}{ccc}
BBP per capita lopende dollar & Wisselmarkt & PPP \\ \hline
Verenigde Staten & 54.629 & 54.629 \\
Belgi\"e & 47.517 & 42.725 \\
Eurozone & 39.567 & 38.694 \\
Brazili\"e & 11.385 & 15.838 \\
Indi\"e & 1.596 & 5.708 \\
Burkina Faso & 713 & 1.668
\end{tabular}
\caption{De BBP's per capita vergeleken}
\label{tab:h5bbpcap}
\end{table}

\subsubsection{Inflatie}

Wij gaan er vandaag de dag van uit dat de prijzen steeds stijgen. Men noemt dat \entry{inflatie}. Maar dit is niet altijd het geval geweest. Kijk maar naar figuur \ref{fig:h5inflatie}. Inflatie begon pas aan het einde van de 19\^{e} eeuw\footnote{De romans van Jane Austin waren van die tijd, en verwijzen naar vermogens en inkomens van een bepaald bedrag. Voor de lezers van haar boeken waren de bedragen veelzeggend, omdat er geen inflatie was. Ze gaven de koopkracht weer.}.

\begin{figure}[H]
\small\centering\captionsetup{justification=centering,margin=2cm}
\begin{tikzpicture}
\begin{axis}[ytick={0,0.1,0.2},yticklabels={0\%,10\%,20\%},axis lines=left,axis line style=gray,/pgf/number format/.cd, use comma, width=0.5\linewidth, legend cell align=left,legend style={at={(axis cs:1720,0.12)},anchor=south west,draw=none},ymin=-0.02,ymax=0.20]
\addplot[blue] table [x=Periode, y=Frankrijk, col sep=comma] {Data/H5-Inflatie.csv};
\addplot[red] table [x=Periode, y=Duitsland, col sep=comma] {Data/H5-Inflatie.csv};
\addplot[black] table [x=Periode, y=Verenigde Staten, col sep=comma] {Data/H5-Inflatie.csv};
\addplot[darkgreen] table [x=Periode, y=Verenigd Koninkrijk, col sep=comma] {Data/H5-Inflatie.csv};
\addlegendentry{Frankrijk}
\addlegendentry{Duitsland}
\addlegendentry{Verenigde Staten}
\addlegendentry{Verenigd Koninkrijk}
\end{axis}
\end{tikzpicture}
\caption{De inflatie sinds de Industri\"ele Revolutie}
\label{fig:h5inflatie}
\end{figure}

De inflatie meet men aan de hand van een \term{prijsindex}. Deze meet de prijsstijging tijdens een bepaalde periode. Specifiek gebruikt men de \entry{BBP-deflator}. De deflator is verschillend van de \term{consumptieprijsindex}, die zich beperkt tot de prijsstijging van de consumptiegoederen.
\par De BBP-deflator is \textit{impliciet}, omdat hij gekend is als het nominaal - en het re\"eeel BBP gegeven zijn. Hij is immers gelijk aan het nominaal - gedeeld door het re\"eel BBP, vermenigvuldigd met 100. In feite is de BB-deflator een bijproduct van het berekenen van het re\"ele BBP via kettingindexen.\\

\par\noindent Bij de consumptieprijsindex (van Laspeyres\footnote{Ernst Louis \'Etienne Laspeyres, Duits econoom.}) is de techniek anders. Zoals eerder opgemerkt is deze gelijk aan :
$$P_t=\frac{\sum_{i=1}^k p_i^{t}\cdot q_i^{t}}{\sum_{i=1}^k p_i^{0}\cdot q_i^{t}}\times 100$$
Hier worden de prijzen van het huidige jaar en de kwantiteiten van het basisjaar gewaardeerd tegen de prijzen en kwantiteiten van het basisjaar.
\par Deze index is belangrijk omdat onze lonen ge\"indexeerd worden. Zo'n loonindexering zorgt ervoor dat als de prijzen stijgen, de nominale lonen meestijgen, zodanig dat er geen re\"ele loonverlies is.\\

\par Concreet wordt de Laspeyres-index berekend als volgt :
$$P_t=\sum_{i=1}^k w_i^0\cdot\frac{p_i^t}{p_i^0}\times 100=\frac{\sum_{i=1}^k \frac{p_i^t}{p_t^0}\cdot p_i^{0}\cdot q_i^{0}}{\sum_{i=1}^k p_i^{0}\cdot q_i^{t}}\times 100$$
In de teller staat wat wij uitgeven \textit{aan \'e\'en product}, en in de noemer hoeveel we uitgeven \textit{aan alle producten samen}. Samen geeft dat een \term{wegingsco\"effici\"ent} $w_i^0$ van een product $i$. Het is het aandeel van dat product $i$ in het budget van het basisjaar.
\par De Laspeyres-prijsindex kan daarom geformuleerd worden als \textit{de som van de individuele prijsindexen gewogen met het budgetaandeel in het basisjaar}.
\par In Belgi\"e weegt de huisvesting, dan de voeding, en dan de gezondheidszorg het meeste door.\\

\par Bij dreiging van \entry{deflatie}, d. i. een daling van de prijzen zal de centrale bank aan \entry{expansief monetair beleid} doen. Dit wil zeggen dat er veel geld in omloop wordt gebracht en de interesten laag worden gehouden.