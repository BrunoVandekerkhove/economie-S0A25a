% Inleiding
\newentry{micro-economie}{een onderdeel van de economische wetenschap waarbij men bestudeert hoe individuele economische agenten in een economie beslissingen nemen over de allocatie van schaarse goederen, meestal in markten waar goederen of diensten worden gekocht en verkocht. Ze onderzoekt tevens hoe deze beslissingen en gedragingen van invloed zijn op vraag en aanbod van goederen en diensten, hoe prijzen tot stand komen en hoe deze prijzen, op hun beurt, weer vraag en aanbod van goederen en diensten bepalen}
\newentry{macro-economie}{het onderdeel van de economische wetenschap dat zich bezighoudt met de bestudering van economische systemen als geheel. Meestal zal de macro-economie de economische samenhangen in een land (de `volkshuishouding') bestuderen. In de macro-economie bestudeert men geaggregeerde economische grootheden om op die manier te begrijpen hoe de economie in zijn geheel functioneert}
\newentrystyled{reele sector}{re\"ele sector}{bestaat uit de totale vraag naar goederen van de consumenten, bedrijven, overheid en het buitenland}
\newentrystyled{financiele sector}{financi\"ele sector}{de bedrijven en instellingen die diensten leveren met betrekking op financi\"ele producten (zoals banken, verzekeraars, ...)}
\newentry{marginale kost}{de kost die \'e\'en extra product met zich meebrengt}
\newentry{marginale opbrengst}{de opbrengst die \'e\'en extra product met zich meebrengt}
\newentry{afgeleide}{(het differentiaalquoti\"ent) een maat voor de verandering die een functie ondergaat als de argumenten van deze functie een infinitesimaal kleine verandering ondergaan}
\newentry{homo economicus}{een mensbeeld waarin de mens gericht is op bevrediging van zijn behoeften op effici\"ente, rationele of logische wijze}
\newentry{intrinsieke motivatie}{motivatie vanuit de persoon zelf}
\newentry{extrensieke motivatie}{motivatie die ontstaat vanuit een externe bron}

% Waarover en Hoe Denken Economen
\newentrystyled{bbp}{Bruto Binnenlands Product (BBP)}{de som van de gedurende \'e\'en jaar door alle binnenlandse productie-eenheden toegevoegde waarden}
\newentry{tijdsreeks}{gegevens die verzameld zijn in de loop van een bepaalde periode}
\newentry{doorsnedegegevens}{metingen op een bepaald moment over verschillende entiteiten}
\newentry{toegevoegde waarde}{het verschil tussen de marktwaarde van productie en de daarvoor ingekochte grondstoffen}
\newentry{intermediaire goederen}{(of half-afgewerkte goederen) zijn goederen die \'e\'en of meerdere industri\"ele transformaties hebben ondergaan en die nog moeten worden omgezet in afgewerkte (`finale') goederen}
\newentrystyled{reele groei}{re\"ele groei}{economische groei waarbij de inflatie en prijsstijgingen zijn verdisconteerd in de berekening daarvan}
\newentry{nominale groei}{economische groei die zowel het gevolg is van prijsverandering als van re\"ele groei}
\newentry{koopkrachtpariteit (KKP)}{\textit{(in het Engels : Purchasing Power Parity dollars)} een manier om de koopkracht van twee regio's te vergelijken}
\newentry{convergentie}{wanneer landen met een relatief laag BBP per capita relatief sneller groeien dan landen met een relatief hoog BBP per capita}
\newentry{divergentie}{wanneer landen met een relatief laag BBP per capita relatief trager groeien dan landen met een relatief hoog BBP per capita}
\newentry{arbeidsproductiviteit}{de hoeveelheid productie die per tijdseenheid wordt geproduceerd door \'e\'en arbeider. Vaak wordt het ook gemeten door per arbeider de toegevoegde waarde te nemen}
\newentry{vergrijzing}{een term die gebruikt wordt om aan te geven dat het aandeel van ouderen in de bevolking stijgt en daardoor een stijging van de gemiddelde leeftijd veroorzaakt}
\newentry{activiteitsgraad}{de mate waarin de bevolking zich op actieve leeftijd aanbiedt op de arbeidsmarkt (met andere woorden, een job heeft of zoekt)}
\newentry{werkloosheidsgraad}{het procentuele aandeel van de werklozen binnen de beroepsbevolking}
\newentry{werkgelegenheidsgraad}{de verhouding tussen het aantal werkenden en de bevolking in de beroepsactieve leeftijd}
\newentry{arbeidsduur}{de tijd gedurende welke het personeel ter beschikking is van de werkgever}
\newentry{arbeidsverdeling}{het opsplitsen van taken of arbeid}
\newentry{internationale handel}{handel tussen landen}
\newentry{economische kringloop}{een schematisch model van de werking van de economie als systeem. Deze kringloop is een abstracte voorstelling van de relaties tussen de gezinnen en de bedrijven in een land}
\newentry{opportuniteitskost}{de waarde van de verloren gegane best mogelijke alternatieve aanwending van de gebruikte schaarse middelen}
\newentry{comparatief voordeel}{als een land een bepaald product relatief goedkoper kan produceren dan een handelspartner in vergelijking met andere producten, zelfs wanneer \'e\'en speler alles voordeliger kan produceren dan de andere speler. De theorie wordt over het algemeen toegeschreven aan David Ricardo}
\newentry{zero sum game}{een spel waarbij de opbrengst een constante waarde heeft. Als een speler wint moeten de andere spelers evenveel verliezen. Er is dus maar \'e\'en buit te verdelen}
\newentry{transformatiecurve}{een grafiek die de productieratio's vergelijkt van twee commodities, waarvoor in het productieproces dezelfde vaste hoeveelheid productiefactor worden gebruikt}
\newentry{stroomveranderlijke}{een veranderlijke die betrekking heeft op een periode, zoals het inkomen, de productie, bestedingen, opbrengsten, kosten en winst. Het is een variabele die gemeten is als een hoeveelheid per tijdseenheid}
\newentry{voorraadveranderlijke}{een variabele die gemeten wordt als een hoeveelheid op een bepaald punt (zoals het vermogen, de bezittingen, de schulden, ...}
\newentry{econometrie}{de discipline binnen de economische wetenschap die zich richt op het kwantificeren van de relaties tussen economische grootheden}
\newentry{causaliteit}{als gebeurtenissen plaatsvinden als gevolg van bepaalde andere gebeurtenissen die daaraan voorafgegaan zijn}
\newentry{correlatie}{de statistische samenhang tussen twee grootheden}
\newentry{positieve uitspraak}{een uitspraak die \'of juist, \'of fout is. Een positieve uitspraak is dus weerlegbaar}
\newentry{normatieve uitspraak}{schrijft voor hoe iets zou moeten zijn. Een normatieve uitspraak is niet weerlegbaar, hij volgt uit de normen en waarden van degene die de uitspraak doet}
\newentry{mediaaninkomen}{het mediaan inkomen ligt halverwege de inkomensverdeling, dat wil zeggen dat de helft van de inkomenstrekkers meer verdient dan het mediaan inkomen en de andere helft minder}
\newentry{werkintensiteit}{}

% Vraag & Aanbod
\newentry{prijsnemerschap}{als de prijs van een product gegeven is en niet be\"invloed wordt door het gedrag van consumenten (of producenten). Als je in een horeca-bedrijf koffies koopt, dan zal de prijs van koffie niet veranderen}
\newentry{homogeen}{van dezelfde aard ; homogene producten zijn perfect inwisselbaar, verschillen niet in kwaliteit}
\newentry{algemene vraagfunctie}{toont het verband aan tussen de gevraagde hoeveelheid van een bepaald goed en de prijs ervan}
\newentry{substitutieproduct}{een product dat een ander product gedeeltelijk of volledig kan vervangen. Rijst en aardappelen zijn een voorbeeld van substituten}
\newentry{complementair product}{een product dat een ander product aanvult (waarvan het gebruik dus positief gerelateerd is)}
\newentrystyled{partiele vraagfunctie}{parti\"ele vraagfunctie}{een vraagfunctie waarbij men de invloed van \'e\'en factor op de vraag onderzoekt, onder de veronderstelling dat alle andere factoren gelijk blijven}
\newentry{ceteris paribus}{(Latijn, `\textit{overal gelijk blijvend}') wanneer men de invloed van veranderingen in \'e\'en grootheid (de verklarende variabele) op \'e\'en andere grootheid (de te verklaren variabele) onderzoekt}
\newentry{algemene aanbodfunctie}{toont het verband aan tussen de aangeboden hoeveelheid van een bepaald goed en de prijs ervan}
\newentrystyled{partiele aanbodfunctie}{parti\"ele aanbodfunctie}{een aanbodfunctie waarbij men de invloed van \'e\'en factor op het aanbod onderzoekt, onder de veronderstelling dat alle andere factoren gelijk blijven}
\newentry{marktevenwicht}{wanneer vraag en aanbod op een bepaalde markt in evenwicht zijn. In een marktevenwicht zijn vraag en aanbod aan elkaar gelijk}
\newentry{vraagoverschot}{als er bij een bepaalde prijs meer goederen of diensten worden aangeboden dan er wordt gevraagd}
\newentry{aanbodoverschot}{als er bij een bepaalde prijs meer goederen of diensten worden gevraagd dan er wordt aangeboden}
\newentry{consumentensurplus}{het cumulatieve verschil tussen de individuele gebruikswaarde van een consumptiegoed en de marktprijs. Het is dus wat de mensen bereid waren m\'e\'er te betalen, maar dus niet hoeven te betalen}
\newentry{producentensurplus}{het verschil tussen de evenwichtsprijs en de reservatieprijs (de laagste prijs waartegen de producenten bereid zijn te verkopen). Het is dus wat de consumenten m\'e\'er betalen dan wat de producent er minstens voor zou willen vragen}
\newentry{reserveringsprijs}{de hoogste prijs die een koper bereid is voor goederen of diensten te betalen, of de laagste prijs waartegen een verkoper bereid is om een goed of dienst te verkopen}
\newentrystyled{pareto-efficientie}{Pareto-effici\"entie}{een economie is Pareto-effici\"ent als iedere verandering in die economie, die voor de ene een welvaartsverbetering teweeg brengt, tegelijkertijd een welvaartsverlies voor iemand anders impliceert. Het Pareto-optimum ligt op de productiemogelijkhedencurve}
\newentry{boogelasticiteit}{een maat voor de verandering van een (te verklaren) variabele wanneer een (verklarende) variabele met een bepaalde hoeveelheid verandert. De veranderingen beschrijven een `boog' van de overeenkomstige functie, vandaar de naam}
\newentry{puntelasticiteit}{een maat voor de verandering van een (te verklaren) variabele wanneer een (verklarende) variabele infinitesimaal verandert}
\newentry{normaal goed}{een goed waarvan de consumptie toeneemt als het inkomen toeneemt (de inkomenselasticiteit is dus positief)}
\newentry{inferieur goed}{een goed waarvan de consumptie afneemt als het inkomen toeneemt (de inkomenselasticiteit is dus negatief)}
\newentry{noodzakelijk goed}{een goed waarvan de consumptie toeneemt als het inkomen toeneemt, maar waarvan het budgetaandeel dan wel daalt ($0<\epsilon_y^V<1$)}
\newentry{luxegoed}{een goed waarvan de consumptie afneemt als het inkomen toeneemt, en waarvan het budgetaandeel dan steeds stijgt ($1<\epsilon_y^V$)}
\newentry{kruiselingse prijselasticiteit}{een maat voor de verandering van de vraag naar een product bij prijsverandering van een ander product}
\newentry{vraagschok}{een plotselinge gebeurtenis die al of niet tijdelijk leidt tot stijgingen of dalingen van de vraag naar goederen of diensten. Een positieve vraagschok verhoogt de vraag en een negatieve vraagschok leidt tot een daling van de vraag}
\newentry{aanbodschok}{een plotselinge gebeurtenis die al of niet tijdelijk leidt tot stijgingen of dalingen van het aanbod naar goederen of diensten. Een positieve aanbodschok verhoogt het aanbod en een negatieve aanbodschok leidt tot een daling van het aanbod}
\newentry{belasting}{een algemene, verplichte betaling aan de overheid door een rechtssubject, waartegenover geen individuele prestatie van die overheid aan dat rechtssubject staat}
\newentry{subsidie}{een tijdelijke bijdrage van de overheid of een niet-commerci\"ele organisatie ten behoeve van het starten van een activiteit waarvan het economische belang niet direct voor de hand ligt}
\newentry{marktconform}{een maatregel is marktconform als hij het marktmechanisme intact laat. Belastingen zijn marktconform}
\newentry{bindende maximumprijs}{een door de overheid opgelegde prijs waarboven niet mag worden verkocht}
\newentry{bindende minimumprijs}{een door de overheid opgelegde prijs waaronder niet mag worden verkocht}
\newentry{rantsoenering}{een noodzakelijke en eerlijke verdeling van alle goederen over de bevolking bedoeld}
\newentry{steunaankoop}{aankoop van de eigen goederen door de overheid of centrale bank - alles wat geen afzet vindt op de private markt wordt aangekocht}

% Marktvormen
\newentry{marktvorm}{een wijze van ontmoeting tussen vragers en aanbieders. De meest gekende marktvormen zijn volmaakte mededinging en het monopolie}
\newentry{volmaakte mededinging}{een model van een markt met onbelemmerde marktwerking. Er zijn veel kopers en verkopers, de producten moeten perfect inwisselbaar (homogeen) zijn, en alle marktpartners zijn perfect ge\"informeerd}
\newentry{creatieve destructie}{een proces van voortdurende innovatie, waarbij succesvolle toepassingen van nieuwe technieken de oude vernietigen. De term is afkomstig van de socioloog en econoom Werner Sombart}
\newentry{monopolie}{een marktvorm met juist \'e\'en aanbieder van een producten dat geen goede substituten heeft. Deze producent verkeert in een positie van macht}
\newentry{prijsdiscriminatie}{het vragen van verschillende prijzen voor gelijke producten aan verschillende afnemersgroepen}
\newentry{perfecte prijsdiscriminatie}{wanneer prijsdiscriminatie wordt uitgevoerd door een producent die perfect ge\"informeerd is over de betalingsbereidheid van zijn klanten en hij kan verhinderen dat ze aan elkaar doorverkopen}
\newentry{marktsegmentatie}{hopdelen van een markt in verschillende, van elkaar te onderscheiden groepen afnemers, voor wie het wenselijk kan zijn een bepaalde marketingstrategie toe te passen}
\newentry{nutsbedrijf}{een bedrijf dat, vaak vanuit een monopoliepositie, opereert in een sector die beschouwd wordt zijnde van openbaar nut omdat het belangrijke producten of diensten levert die in het algemeen belang zijn. Hier worden de elektriciteits-, gas- en drinkwatervoorziening toe gerekend}
\newentrystyled{statische efficientie}{statische effici\"entie}{als men de bestaande productiemiddelen (kapitaal en arbeid) optimaal aanwendt - er is dus zowel technische als allocatieve effici\"entie}
\newentrystyled{dynamische efficientie}{dynamische effici\"entie}{gaat over hoe men de optimale uitbreiding van de productiecapaciteit in de hand werkt om de economische groei op lange termijn te bevorderen}
\newentry{patent}{een exclusief (uitsluitend) recht tot het industrieel maken of verkopen van een product of anderszins het exploiteren van een uitvinding}
\newentry{homogeen oligopolie}{een marktvorm met weinig aanbieders (en veel vragers) en een homogeen product}
\newentry{oligopolie}{een marktvorm waarin een economisch product door slechts een paar grote aanbieders wordt aangeboden}
\newentry{kartel}{een overeenkomst tussen bedrijven die bedoeld is om de onderlinge concurrentie te verminderen}
\newentrystyled{cournot-evenwicht}{Cournot-evenwicht}{wanneer de productie van ieder bedrijf de winst van dat bedrijf maximaliseert gegeven de productie van andere bedrijven}
\newentry{speltheorie}{een tak van de wiskunde waarin het nemen van beslissingen centraal staat}
\newentry{dominante strategie}{een strategie bij een spel, een keuze die hoe dan ook de beste is (wat de strategie van de andere spelers ook mag zijn)}
\newentry{gedomineerde strategie}{een strategie die een slechter resultaat oplevert dan alle andere strategie\"en en dus ongeacht de keuze van de andere speler nooit gekozen zal worden}
\newentrystyled{nash-evenwicht}{Nash-evenwicht}{een oplossingsconcept voor een niet-co\"peratief spel, waar twee of meer spelers aan meedoen. In een Nash-evenwicht wordt elke speler geacht de evenwichtsstrategie\"en van de andere spelers te kennen en heeft geen van de spelers er voordeel bij om zijn of haar strategie eenzijdig te wijzigen}
\newentry{productdifferentiatie}{een marketingterm waarmee gedoeld wordt op het aanbrengen van verschillen in het productaanbod van verschillende aanbieders}
\newentry{marktmacht}{de mate waarin die deelnemer zich onafhankelijk van de andere marktspelers kan gedragen. Meestal wordt de term toegepast op aanbieders van goederen of diensten, om de mate aan te duiden waarin de aanbieder een prijs kan vragen die beduidend boven zijn eigen kosten ligt}
\newentry{niche}{een specifiek, vaak klein, afgebakend en bewerkbaar deel van een markt}
\newentry{monopolistische mededinging}{een marktvorm die zowel een kenmerk heeft van een monopolie als van volledige mededinging. Er zijn namelijk veel aanbieders (als bij volmaakte mededinging), terwijl deze aanbieders via marketing hun product enigszins heterogeen maken waardoor merktrouw van de consument kan ontstaan}
\newentry{asymmetrische informatie}{als in een economische transactie de ene partij meer informatie heeft dan de andere partij}
\newentry{averechtse selectie}{betekent eigenlijk ``verkeerde selectie''. Averechtse selectie is het proces dat door asymmetrische informatie de slechte risico's de goede risico's van de markt verdrijven}
\newentry{missing market}{als de vraag naar een goed of dienst bestaat, maar er is geen beschikbaar aanbod van dit gewilde product}
\newentry{moral hazard}{(moreel gevaar of - wangedrag) een economisch begrip dat verwijst naar veranderingen in het gedrag van partijen indien zij niet direct risico lopen voor hun daden}

% Marktfalingen
\newentry{marktfaling}{een situatie waarbij een markt (zelfs in volmaakte mededinging) niet tot Pareto-effici\"entie leidt. Het is een markt waar door \'e\'en of andere omstandigheid het marktevenwicht verschilt van het Pareto-optimum}
\newentry{publiek goed}{een goed dat niet-uitsluitbaar en niet-rivaliserend is. Dit betekent dat je geen vragers kan uitsluiten, en dat de consumptie ervan niemand anders belet om het ook te consumeren}
\newentry{privaat goed}{een goed dat uitsluitbaar en rivaliserend is. Vragers kunnen uitgesloten worden van het gebruik, en de consumptie van het goed belet een andere om het ook te consumeren}
\newentry{extern effect}{een niet gecompenseerde, door derden gemaakte kost of geleden schade als gevolg van een economische activiteit}
\newentry{clubgoed}{een economisch goed waarvan het gebruik niet rivaliserend is, en waarbij individuen uitgesloten kunnen worden}
\newentry{common}{(\textit{common-pool resource}) een goed dat niet uitsluitbaar maar wel rivaliserend is}
\newentry{vrijbuitersgedrag}{het gedrag van een persoon (de vrijbuiter) dat berust op de redenering dat wanneer iedereen geeft, de eigen gift niet meer noodzakelijk is om het publiek goed te kunnen realiseren, met de concludering dat hij net zo goed niets kan geven}
\newentry{eigendomsrechten}{rechten die in sommige landen in de grondwet worden beschermd, waarmee men overeenkomsten kan sluiten, handel kan drijven, een inkomen door arbeid kan verwerven en onroerend goed en priv\'ebezit kan hebben}
\newentry{marktfundamentalisme}{economisch model waarin vraag en aanbod op elk terrein centraal staan en de overheid zo min mogelijk intervenieert}
\newentry{milieuheffing}{een heffing op milieuvervuilende producten of productieprocessen}
\newentrystyled{pigouviaanse belasting}{Pigouviaanse belasting}{een manier voor de overheid om ineffici\"enties die worden veroorzaakt door negatieve externe effecten te corrigeren}
\newentry{emissierecht}{een recht van een land of bedrijf om bepaalde broeikasgassen of andere schadelijke gassen (kooldioxide, methaan, ...) uit te stoten}
\newentry{consumentensoevereiniteit}{het geloof dat de consument best weet wat goed voor hem is}
\newentry{verdienstengoed}{een goedere waarvan de consument de waarde onderschat}
\newentry{demerit good}{een goedere waarvan de consument de waarde overschat}
\newentry{primair inkomen}{het inkomen dat rechtstreeks uit het productieproces voortkomt, waarbij beloningen worden toegekend als tegenprestatie voor de aanbreng van de productiefactoren}
\newentry{beschikbaar inkomen}{het inkomen na aftrek van belastingen plus uitkeringen, dat besteed wordt aan consumptie en besparingen}
\newentry{OESO}{de organisatie voor economische samenwerking en ontwikkeling (Engels : OECD). Het zijn de rijke industrielanden (Europa, VS, Canada, Australie en Japan)}
\newentrystyled{lorenzcurve}{Lorenzcurve}{geeft het verband weer tussen het cumulatief percentage van de bevolkingsomvang, en het cumulatief percentage van de inkomens van diezelfde bevolking}
\newentrystyled{gini-coefficient}{Gini-co\"effici\"ent}{een getal tussen 0 en 1. De waarde 0 correspondeert hierbij met `perfecte gelijkheid' (in dit geval heeft iedereen hetzelfde inkomen) en 1 correspondeert met `perfecte ongelijkheid' (\'e\'en persoon heeft al het inkomen en de rest heeft geen inkomen)}
\newentry{Occupy Wall Street}{een Amerikaanse protestbeweging die in 2011 ontstond. De demonstranten protesteren tegen de hebzucht van Wall Street en andere grote financi\"ele instellingen. De hebzucht in de financi\"ele sector beschouwen zij als een van de belangrijkste oorzaken van de financi\"ele crisis waarin de wereld sinds 2008 terecht is gekomen}
\newentry{assortive mating}{een soort van seksuele selectie waarbij individu's met bepaalde trekken een partner zoeken die dezelfde trekken heeft (`soort zoekt soort'). Rijke mensen trouwen bijvoorbeeld met elkaar}
\newentry{globalisering}{een voortdurend proces van wereldwijde economische, politieke en culturele integratie, met als centraal kenmerk een wereldwijde arbeidsdeling, waarbij productielijnen over de wereld worden gespreid die gedreven worden door de informatie- en communicatietechnologie en door internationale handel}
\newentry{sociale mobiliteit}{de verandering in sociale positie van een persoon of groep binnen de sociale stratificatie. Bij intrageneratiemobiliteit speelt deze stijging of daling op de maatschappelijke ladder zich af in het eigen leven, bij intergeneratiemobiliteit is de sociale positie ten opzichte van de ouders veranderd}
\newentry{armoedegrens}{het inkomen dat iemand nodig heeft om te kunnen voorzien in de basisbehoeften (minimale voorwaarden die nodig zijn om menswaardig te kunnen leven: kleding, goed drinkwater, voldoende voedsel, goede huisvesting, goed onderwijs en goede gezondheidszorg)}
\newentry{progressieve belasting}{een inkomensbelasting die een groter deel van de rijken wegneemt dan van de armen}
\newentry{proportionele belasting}{een inkomensbelasting waarbij ieder inkomen met hetzelfde percentage belast wordt}
\newentry{regressieve belasting}{een inkomensbelasting die een groter deel van de armen wegneemt dan van de rijken}
\newentry{draagkrachtprincipe}{het uitgangspunt dat mensen met een hoger inkomen in verhouding meer belasting betalen dan mensen met een laag inkomen}
\newentry{verzekeringselement}{als, om in aanmerking te komen voor een uitkering, werknemers een bepaald percentage van hun loon betalen als verzekeringspremie. De grootte van de uitkering is gekoppeld aan de grootte van de premie}
\newentrystyled{mattheuseffect}{Mattheuseffect}{het verschijnsel dat rijken ergens meer van profiteren dan armen (referentie naar evangelist Mattheus ; `\textit{aan wie heeft zal worden gegeven}')}
\newentry{pensioen}{een inkomensverzekering waarmee een (gezins)inkomen wordt verzekerd voor wanneer dat wegvalt wegens ouderdom, arbeidsongeschiktheid of overlijden}
\newentry{kapitalisatie}{wanneer ieder betaalt voor zijn eigen pensioen, en later de uitgestelde opbrengst van de eigen inspanning ontvangt. Deze opbrengst wordt uitgehold door inflatie en houdt zeker geen gelijke tred met de algemene welvaartstoename (door aangroei BBP of aangroei verdiensten van de actieven)}
\newentry{repartitieprincipe}{wanneer actieve bijdragen betalen waarmee onmiddelijk de pensioenen van de niet-actieven gefinancierd worden. Dit laat toe de pensioenen voortdurend te verhogen in gelijke tred met de algemene welvaartsstijging}
\newentry{woonbonus}{een systeem van hypotheekrenteaftrek dat in 2005 in Belgi\"e is ingevoerd}
\newentry{huurtoelage}{een maandelijkse tussenkomst in de huur van je kot die je niet moet terugbetalen}
\newentry{efficiency-equity trade-off}{als een gegeven activiteit in een gegeven markt tegelijkertijd effici\"entie verhoogt en rechtvaardigheid verlaagt, of vice versa}
\newentrystyled{ninja-lening}{NINJA-lening}{(`\textit{No Income, No Jobs, No Asset}') een lening gegeven aan individuen zonder inkomen en bezittingen}

% Beschrijvende Macro-Economie
\newentry{loonstarheid}{rigiditeit van de lonen, lonen zijn op de korte termijn star/rigide omdat de lonen vast gelegd zijn in arbeiders vakbonden en daar moeten de werkgevers zich aan houden. Op de lange termijn zijn de lonen wel flexibel want de contracten met de vakbonden kunnen natuurlijk gewijzigd worden}
\newentry{aggregatieve vraag}{de vraag naar alle producten samen}
\newentry{nationale boekhouding}{geeft een kwantitatieve beschrijving van het economisch proces binnen een land, en de economische relaties met het buitenland}
\newentry{kapitaalstock}{de beschikbare door de mens geproduceerde duurzame productiemiddelen (zoals machines, bruggen, wegen, ...)}
\newentry{afschrijving}{het in de boekhouding tot uitdrukking brengen van de waardedaling van een bedrijfsmiddel over een bepaalde periode}
\newentrystyled{bni}{Bruto Nationaal Inkomen (BNI)}{een veelgebruikte maat voor de omvang van een economie. Het is het bruto binnenlands product plus de door de inwoners van het eigen land in het buitenland verdiende primaire inkomens minus de door buitenlanders in het betreffende land verdiende primaire inkomens}
\newentrystyled{ad statistiek}{Algemene Directie Statistiek - Statistics Belgium (AD Statistiek of Statbel)}{\'e\'en van de algemene directies van de (Belgische) Federale Overheidsdienst Economie, K.M.O., Middenstand en Energie. De basisopdracht van de AD Statistiek is het verzamelen, verwerken en verspreiden van betrouwbare cijfergegevens over de Belgische samenleving}
\newentry{investering}{de uitbreiding van de kapitaalgoederenvoorraad}
\newentry{binnenlandse vraag}{de totale voorgenomen economische vraag naar goederen en diensten in een samenleving gedurende een jaar}
\newentry{handelsbalans}{een onderdeel van de betalingsbalans van een land. Aan de ontvangstenkant staat de geldwaarde van de export van een land over een bepaalde periode. Aan de uitgavenkant staat de geldwaarde van de import}
\newentry{besparingspolitiek}{een politieke strategie waarbij men het budget doet inkrimpen. Men verlaagt de binnenlandse vraag}
\newentry{loon-prijsspiraal}{kan ontstaan als gestegen loonkosten per product worden doorberekend in de prijzen, en de aldus ontstane hogere prijzen op hun beurt weer tot hogere looneisen leiden}
\newentry{open economie}{een economie waarin er economische interacties plaatsvinden tussen leden van de binnenlandse gemeenschap (zowel natuurlijke personen als bedrijven) en mensen en bedrijven daarbuiten, die zich dus in een ander land bevinden}
\newentry{factorinkomen}{het door deelname aan de productie gegenereerde inkomen, waarvoor een directe tegenprestatie geleverd wordt en bestaat uit de vier componenten; loon (meestal uit arbeid), huur (de inkomsten uit de exploitatie van onroerend goed) alsmede de vergoeding voor het gebruik van kapitaalgoederen, interest vergoeding op bezit en dividend, de winst uit onderneming}
\newentrystyled{nni}{Netto Nationaal Inkomen (NNI)}{een economische term die het totaal verdiende inkomen van een land in \'e\'en jaar aangeeft}
\newentry{transfer}{een overheveling van koopkracht van het ene land naar het andere land (in feite een cadeau)}
\newentry{lopende rekening}{\'e\'en van de twee primaire componenten van de betalingsbalans. Het is de som van de handelsbalans, het netto factorinkomen en de netto transfers. De lopende rekening is een belangrijke maatstaaf om de buitenlandse handel van een land te duiden}
\newentry{sparen}{het verschil tussen het inkomen en de consumptie}
\newentry{totaal beschikbaar overheidsinkomen}{de som van de netto directe belastingen en de netto indirecte belastingen}
\newentry{reservemunt}{een valuta die door de overheid en andere instanties van een land vaak in grote hoeveelheden achter de hand wordt gehouden als onderdeel van de internationale reserves van een land}
\newentry{betalingsbalans}{een overzicht van de waarde van alle transacties die in een bepaalde periode hebben plaatsgevonden tussen ingezetenen van het land en niet-ingezetenen van andere landen (het buitenland)}
\newentry{recursief}{zichzelf herhalend}
\newentry{deflatie}{een daling van het algemene prijspeil}
\newentry{inflatie}{een stijging van het algemene prijspeil}
\newentry{outputkloof}{(`\textit{output gap}') het verschil tussen het feitelijk - en het potentieel BBP}
\newentry{conjuncturele werkloosheid}{werkloosheid die verband houdt met schommelingen in de economische conjunctuur. In tijden dat het economisch wat minder gaat, worden minder nieuwe werknemers aangenomen, worden contracten niet verlengd, en mensen ontslagen. Ondernemers kunnen zich zelfs gedwongen zien hun bedrijvigheid geheel of gedeeltelijk te be\"eindigen. De totale werkloosheid wordt gevormd door de conjuncturele - en de structurele werkloosheid samen}
\newentry{wisselmarkt}{\'e\'en van de elementen van internationale transactie. Wisselmarkten zijn ontstaan doordat elk land zijn eigen munteenheid gebruikt, daarom zijn internationale transacties ingewikkelder dan binnenlandse. Op de wisselmarkt worden valuta's aangeboden en gevraagd}
\newentry{BBP-deflator}{een maatstaf voor de prijsveranderingen in de economie. Het wordt berekend door het nominaal BBP te delen door het re\"ele BBP}
\newentry{expansief monetair beleid}{beleid van monetaire autoriteiten om de geldhoeveelheid uit te breiden en economische activiteiten te verhogen door met name de interest percentages laag te houden om daarmee het lenen door bedrijven, individuen en banken aan te moedigen}

% Verklarende Macro-Economie
\newentry{natuurlijke werkloosheid}{een vrij theoretische vorm van werkloosheid die zich niet gemakkelijk laat uitleggen. Als iemand een job zoekt, dan zal nog de werkzoekende noch de werkgever zomaar de eerste de beste opportuniteit of kandidaat kiezen. Daardoor is er even werkloosheid. Bij desindustrialisatie zou het ook kunnen dat er handarbeiders werk zoeken terwijl de vacatures andere profielen vraagt. De natuurlijke werkloosheid is uiteindelijk een werkloosheid die blijft bestaan, zelfs al draait de economie op volle toeren}
\newentry{scholingspremie}{het extra loon dat je krijgt omdat je langer gestudeerd hebt}
\newentry{brutoloon}{het loon dat de werkgever betaalt aan de werknemer}
\newentry{nettoloon}{het brutoloon min de werkgeversbijdragen}
\newentry{loonkosten}{de som van het brutoloon en de sociale bijdragen}
\newentry{loonwig}{het verschil bedoeld tussen de loonkosten voor de werkgever en het nettoloon dat de werknemer ontvangt}
\newentrystyled{efficientieloon}{effici\"entieloon}{een loon dat boven het evenwicht zit}
\newentry{frictionele werkloosheid}{kortdurige werkloosheid die voortvloeit uit het feit dat het tijd kost eer werknemers een baan gevonden hebben die best bij hun vaardigheden en voorkeuren past}
\newentry{structurele werkloosheid}{werkloosheid die geen verband houdt met schommelingen in de economische conjunctuur. Oorzaken zijn bijvoorbeeld het niet aansluiten van arbeidsvraag en arbeidsaanbod, veranderingen in het productieproces (waardoor minder arbeid nodig is), ...}
\newentry{productiefunctie}{een functie die de output van een bedrijf, een bedrijfstak of een gehele economie specifieert voor alle combinaties van productiefactoren (inputs)}
\newentrystyled{tfp}{totale factorproductiviteit (TFP)}{een variabele die het effect van de groei van arbeid en kapitaal op de totale productiviteitsgroei voorstelt. Het is een maat voor de effici\"entie met dewelke de productiefactoren worden ingezet}
\newentry{kapitaalintensiteit}{de verhouding tussen de ingeschakelde hoeveelheid kapitaal en de ingeschakelde hoeveelheid arbeid in het productieproces}

% Appendix
\newentry{logaritme}{geeft voor een bepaald getal de exponent waarmee een constante waarde (het `grondtal') moet worden verheven om dat bepaalde getal als resultaat te bekomen. De `natuurlijke logaritme' (vaak genoteerd als $ln(x)$) heeft als grondtal $e=2,718281828...$}
\newentry{groeivoet}{de procentuele verandering van het bruto binnenlands product tegenoever het voorgaande jaar}
\newentry{pareto-verbetering}{wanneer een verandering wordt doorgevoerd waarbij minstens \'e\'en individu van een groep er voordeel uit haalt, zonder dat \'e\'en van de anderen uit de groep er door benadeeld wordt}
\newentry{schaalopbrengsten}{de extra opbrengsten als men alle productiefactoren met een extra eenheid laat toenemen}

% Make Glossary
\makeglossaries

% Cross Entries
\newcrossentry{econ}{homo economicus}{}
\newcrossentry{goederenmarkt}{reele sector}{}
\newcrossentry{participatiegraad}{activiteitsgraad}{}
\newcrossentry{werkzaamheidsgraad}{werkgelegenheidsgraad}{}
\newcrossentry{wereldhandel}{internationale handel}{}
\newcrossentry{productiemogelijkhedencurve}{transformatiecurve}{}
\newcrossentry{nulsomspel}{zero sum game}{}
\newcrossentry{natuurlijke logaritme}{logaritme}{de logaritme met grondtal $e=2.718281828...$}
\newcrossentry{algemene vraagvergelijking}{algemene vraagfunctie}{}
\newcrossentry{algemene aanbodvergelijking}{algemene vraagfunctie}{}
\newcrossentry{natuurlijk monopolie}{monopolie}{een monopolie dat zich voordoet als er bij de productie van een bepaald goed schaalvoordelen blijven optreden, welke omvang de productie ook heeft. Hoe meer een bedrijf uitbreidt, hoe meer de gemiddelde en marginale kosten dalen}
\newcrossentry{marginal cost pricing}{bindende maximumprijs}{een bindende maximumprijs waarbij de prijs overeenkomt met de marginale kost}
\newcrossentry{octrooi}{patent}{}
\newcrossentry{duopolie}{oligopolie}{een oligopolie waarbij er slechts twee aanbieders zijn}
\newcrossentry{emissiehandel}{emissierecht}{de handel in emissierechten}
\newcrossentry{merit good}{verdienstengoed}{}
\newcrossentry{vlaktaks}{proportionele belasting}{}
\newcrossentry{fondsvorming}{kapitalisatie}{}
\newcrossentry{depreciatie}{afschrijving}{}
\newcrossentrystyled{statbel}{Statbel}{ad statistiek}{}
\newcrossentrystyled{nis}{Nationaal Instituut voor de Statistiek (NIS)}{ad statistiek}{het voormalige AD Statistiek}
\newcrossentry{austeriteitspolitiek}{besparingspolitiek}{}
\newcrossentrystyled{nnbi}{Netto Nationaal Beschikbaar Inkomen (NNBI)}{nni}{de som van het netto nationaal inkomen en de netto transfers}
\newcrossentry{inflatoire druk}{inflatie}{ontwikkelingen die het prijspeil omhoog drukken, zoals loonstijgingen die de productiviteitsstijgingen overtreffen}
\newcrossentry{desinflatie}{inflatie}{een daling van de inflatie}